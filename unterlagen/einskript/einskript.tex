\documentclass[a4paper,12pt]{article}



\usepackage{graphicx}
\usepackage{../../exam2e}

\usepackage{hyperref}

\begin{document}

\section{Prüfungsaufgaben}


\begin{question}
	Eine Frage mit der Nummer \thequestion
\begin{parts}
	\part Teilaufgabe \\ Lorem ipsum\ldots
	\begin{subparts}
		\subpart Unteraufgabe mit Nummer
	\end{subparts}
	\part Nächste Teilaufgabe lautet berechne:
\begin{equation}
	\aaa\quad \frac{1}{2} \qquad\qquad
	\aaa\quad \frac{1}{2} 
\end{equation}
\end{parts}
\end{question}% Ende von Aufgabe 1

\begin{question}
 	
Eine zweite Frage. Hier folgt als nächste Struktur nicht die Teilaufgabe, sondern gleich die Unteraufgabe. Die Abstände werden automatisch angepasst.
\begin{subparts}
		\subpart Unteraufgabe. Untersuchen Sie  dazu:
	\begin{subsubparts}
		\subsubpart lorem ipsum
		\subsubpart lorem ipsum
	\end{subsubparts}
		\subpart Unteraufgabe. Untersuchen Sie  dazu:
	\begin{subsubparts}
		\subsubpart lorem ipsum
		\subsubpart lorem ipsum
	\end{subsubparts}
\end{subparts}
\end{question}




\begin{question}
	Diese Aufgabe zeigt die ganze Struktur der documentclass \texttt{exam2e}, also question, part, subpart, subsubpart und choice:
\begin{parts}
	\part Teilaufgabe
\begin{subparts}
	\subpart Unteraufgabe
	\begin{subsubparts}
		\subsubpart Unterunteraufgabe
	\begin{checkboxes}
		\choice Ankreuzmöglichkeit (alias choice)
	\end{checkboxes}
	\end{subsubparts}
\end{subparts}
	\part Und dann wählen Sie noch hier:
	\begin{checkboxes}
		\choice Ankreuzmöglichkeit 1 \\ mit einem Zeilenumbruch.
		\choice Ankreuzmöglichkeit 2
	\end{checkboxes}
\end{parts}
\end{question}

\end{document}













