%% TRUEFALSES 

% stellt die Umgebung ''truefalses'' sowie den Befehl
%   \truefalse
% bereit, um Ankreuzaufgaben im folgenden Format zu TeXen:
% Kreuzen Sie jeweils an, ob die Aussage wahr (W) oder falsch (F) ist:
% w  f 
% ◻  ◻  Aussage 1 über Sachverhalt
% ◻  ◻  Aussage 2 über Sachverhalt usw. 



% Dasselbe interface wie exams \tl mit den Optionen T und F:
% \tl[T] oder \tl[F] oder einfach \tl (ohne Lösung im Quellcode)
\newcommand*{\truefalse}[1][{}]{% #1 Voreinstellung ist {}
\@ifundefined{ifprintanswers}{%
    \item[{\makebox[0pt][l]{$\square$}\hspace{2.5em}\makebox[0pt][r]{$\square$}}]
}{ 
\ifprintanswers% exam's Schalter für die Lösungen
    \ifthenelse{\equal{#1}{T}}{%
        \item[{\makebox[0pt][l]{$\checkmark$}\hspace{2.5em}\makebox[0pt][r]{$\square$}}]
    }{% sonst prüfe ob gleich F:
    \ifthenelse{\equal{#1}{F}}{%
        \item[{\makebox[0pt][l]{$\square$}\hspace{2.5em}\makebox[0pt][r]{$\checkmark$}}]
    }{% ansonsten ist eine Antwort nicht angegeben:
        \item[{\makebox[0pt][l]{$\square$}\hspace{2.5em}\makebox[0pt][r]{$\square$}}]
    }%
    }%
\else% sonst (sprich Antworten werden alle nicht gedruckt)
        \item[{\makebox[0pt][l]{$\square$}\hspace{2.5em}\makebox[0pt][r]{$\square$}}]
\fi
}
}% end newcommand


% Sub-Routinen für die Beschriftung der Spalten mit
% W  F (falls aufgerufen mit \begin{truefalses}[WF] )
% T  F (falls aufgerufen mit \begin{truefalses}[TF] )
% Mit dieser Definition darf die Beschriftung kein @ enthalten.
\def\extracttfletter@i#1#2@{\makebox[0pt][l]{\footnotesize{}#1}}
\def\extracttfletter@ii#1#2@{\makebox[0pt][r]{\footnotesize{}#2\,}}

% Umgebung für die Aufrufe von \truefalse:
\newenvironment{truefalses}[1][WF]{% #1 Voreinstellung ist "WF"
\vspace*{-1.2ex}% weniger vertikaler Abstand vor der ersten Zeile
\begin{list}{{}}{%
    \setlength{\leftmargin}{3.5em}% linke Einrückung
    \setlength{\labelwidth}{2.5em}% Länge der Beschriftung (vgl. \truefalse)
    % Abstand zwischen Beschriftung und Text:
    \setlength{\labelsep}{1em}% Voreinstellung ist .5em%
    }% Ende der Definition der Liste.
    % Füge eine Zeile ein mit der Beschriftung der Spalten für die Kreuze:
    \item[{\extracttfletter@i#1@{}%
        \hspace{2.5em}%
        \extracttfletter@ii#1@{}%
        }]%
        \phantom{.}% Die Zeile darf nicht leer sein.
    \vspace*{-1ex}% weniger vertikaler Abstand nach der Beschriftung der Spalten.
}{%
    \end{list}
}% Ende Definition der Umgebung 'truefalses'.


