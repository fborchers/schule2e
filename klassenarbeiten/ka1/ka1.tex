\documentclass{../../exam2e}

\usepackage{graphicx}

\begin{document}

%\klassenarbeitszeile{Klassenarbeit 1}{Mathematik}{10c}{10.10.2023}

\begin{klassenarbeitskopf}
	Thema:\\
	Zeit:	
\end{klassenarbeitskopf}


\begin{questions}
\begin{question}
	A question (numbered \thequestion)
\begin{parts}
	\part Part \thepartno\ with counter at \arabic{partno}.\\
	lorem ipsum
	\begin{subparts}
		\subpart Subpart (labelled \thesubpart)
	\end{subparts}
	\addtocounter{partno}{9}
	\part lorem ipsum
\end{parts}
\end{question}% end question 1.

\question A second question
\begin{subparts}
	\subpart Subpart (labelled \thesubpart)
	\begin{subsubparts}
		\subsubpart lorem ipsum
	\end{subsubparts}
\end{subparts}

\clearpage

\question A third question
\begin{subsubparts}
	\subsubpart Subsubpart (labelled \thesubsubpart)
\end{subsubparts}

\question A forth question
\begin{subparts}
	\subpart[4] Subpart (labelled \thesubpart)
	\begin{itemize}
		\item lorem ipsum
		\item lorem ipsum
	\end{itemize}
\end{subparts}


\end{questions}

\clearpage
\addtocounter{page}{-1}
\thispagestyle{empty} 
\fillwithgrid{\stretch{1}}  % Karos

%% Leere Seite mit Zeilen ---
%\clearpage
%\addtocounter{page}{-1}
%\thispagestyle{empty} 
%\fillwithgrid{\stretch{1}}  % Karos
%\fillwithlines{\stretch{1}} % Zeilen

%% Add N cm of empty lines to page ---
% N cm Zeilen, e.g. 5cm:
% \fillwithgrid{Ncm}  % Karos
% \fillwithlines{Ncm} % Zeilen

%% Fill empty lines till end of page ---
% \fillwithgrid{\stretch{1}}  % Karos
% \fillwithlines{\stretch{1}} % Zeilen

\end{document}