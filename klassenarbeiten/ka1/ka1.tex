\documentclass{../../exam2e}

\usepackage{graphicx}

\begin{document}

%\klassenarbeitszeile{Klassenarbeit 1}{Mathematik}{10c}{10.10.2023}

\begin{klassenarbeitskopf}{Klassenarbeit 1}{Mathematik}{10c}{10.10.2023}
	Thema:\\
	Zeit:	
\end{klassenarbeitskopf}


\begin{questions}

\begin{question}
	Eine Frage mit der Nummer \thequestion
\begin{parts}
	\part Teilaufgabe~\thepartno\ mit dem \TeX{}-Counter bei~\arabic{partno}.\\ Lorem ipsum\ldots
	\begin{subparts}
		\subpart Unteraufgabe mit Nummer~\arabic{subpart}
	\end{subparts}
	\part Nächste Teilaufgabe
\end{parts}
\end{question}% end question 1.

\question Eine zweite Frage. Hier folgt als nächste Struktur nicht die Teilaufgabe, sondern gleich die Unteraufgabe. Die Abstände werden automatisch angepasst.
\begin{subparts}
		\subpart Unteraufgabe mit Nummer~\arabic{subpart}. Untersuchen Sie  dazu:
	\begin{subsubparts}
		\subsubpart lorem ipsum
		\subsubpart lorem ipsum
	\end{subsubparts}
\end{subparts}








\end{questions}

\clearpage
\addtocounter{page}{-1}
\thispagestyle{empty} 
\fillwithgrid{\stretch{1}}  % Kästchen

%% Leere Seite mit Zeilen/Kästchen zum Schreiben ---
%\clearpage
%\addtocounter{page}{-1}
%\thispagestyle{empty} 
%\fillwithgrid{\stretch{1}}  % Kästchen
%\fillwithlines{\stretch{1}} % Zeilen

%% Füge N cm leerer Zeilen/Kästchen zum Schreiben ein ---
% \fillwithgrid{Ncm}  % Kästchen, z.B. 5cm
% \fillwithlines{Ncm} % Zeilen, z.B. 5cm

%% Füge Zeilen/Kästchen bis zum Seitenende ein ---
% \fillwithgrid{\stretch{1}}  % Kästchen
% \fillwithlines{\stretch{1}} % Zeilen

\end{document}