\documentclass[12pt]{../exam2e}

\usepackage{../gess}
\usepackage[english,ngerman]{babel}


%% Seitenlayout (mit 'exam') --------
\extrawidth{.5in}
\extraheadheight[-0.8in]{-.7in}% optionales Arg. trifft auf erste Seite zu.
\extrafootheight{-.2in}

% oder alternativ stattdessen Seitenlayout mit 'geometry'
% \usepackage{geometry}
% \geometry{%
% 	margin=1.5cm
% }


\begin{document}


% Dokumentkopf --------
\thispagestyle{empty}
%\klassenarbeitszeile{Klassenarbeit 1}{Mathematik}{10c}{10.10.2023}
\begin{klassenarbeitskopf}{Klassenarbeit 1}{Mathematik}{10c}{10.10.2023}
	Thema:\\
	Zeit:	
\end{klassenarbeitskopf}


\begin{questions}% Fragen --------



\begin{question}[3]
	Eine Frage mit der Nummer \thequestion
\begin{parts}
	\part Teilaufgabe \\ Lorem ipsum\ldots
	\begin{subparts}
		\subpart Unteraufgabe mit Nummer
	\end{subparts}
	\part Nächste Teilaufgabe
\end{parts}
\end{question}% Ende von Aufgabe 1

\question[2] Eine zweite Frage. Hier folgt als nächste Struktur nicht die Teilaufgabe, sondern gleich die Unteraufgabe. Die Abstände werden automatisch angepasst.
\begin{subparts}
		\subpart Unteraufgabe. Untersuchen Sie  dazu:
	\begin{subsubparts}
		\subsubpart lorem ipsum
		\subsubpart lorem ipsum
	\end{subsubparts}
\end{subparts}



\clearpage

\renewcommand{\thesubpart}{\arabic{subpart}}% Unteraufgaben nummeriert
\renewcommand{\subpartlabel}{\thequestion.\thepartno.\thesubpart}% als: 1.1.1
\qformat{\bfseries Aufgabe~\thequestion\hfill}
\question
Für die Abituraufgaben kann der Zähler und Punkte auch angepasst werden wie in dieser Aufgabe: 
\begin{parts}
\part[1] Teilaufgabe
\begin{subparts}
	\subpart Unteraufgabe
\end{subparts}
\pointsdroppedatright
	\part[2] Weitere Teilaufgabe. Die Punkte werden mit dem Befehl \textbackslash droppoints am Ende des Absatzes gedruckt. 
	\droppoints
\end{parts}



\end{questions}% Ende der Fragen --------


\clearpage
\addtocounter{page}{-1}
\thispagestyle{empty} 
\fillwithgrid{\stretch{1}}  % Kästchen

%% Eine leere Seite mit Zeilen/Kästchen zum Schreiben ---
%\clearpage
%\addtocounter{page}{-1}
%\thispagestyle{empty} 
%\fillwithgrid{\stretch{1}}  % Kästchen
%\fillwithlines{\stretch{1}} % Zeilen

%% Füge N cm leerer Zeilen/Kästchen zum Schreiben ein ---
% \fillwithgrid{Ncm}  % Kästchen, z.B. 5cm
% \fillwithlines{Ncm} % Zeilen, z.B. 5cm

%% Füge Zeilen/Kästchen bis zum Seitenende ein ---
% \fillwithgrid{\stretch{1}}  % Kästchen
% \fillwithlines{\stretch{1}} % Zeilen

\end{document}