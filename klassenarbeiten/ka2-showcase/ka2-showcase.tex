\documentclass[a4paper,11pt]{../exam2e}

\usepackage{graphicx}
\usepackage[ngerman]{babel}

\usepackage{amssymb}

% Für die Ankreuzaufgabe wahr/falsch lade die Definitionen aus:
%% TRUEFALSES 

% stellt die Umgebung ''truefalses'' sowie den Befehl
%   \truefalse
% bereit, um Ankreuzaufgaben im folgenden Format zu TeXen:
% Kreuzen Sie jeweils an, ob die Aussage wahr (W) oder falsch (F) ist:
% w  f 
% ◻  ◻  Aussage 1 über Sachverhalt
% ◻  ◻  Aussage 2 über Sachverhalt usw. 



% Dasselbe interface wie exams \tl mit den Optionen T und F:
% \tl[T] oder \tl[F] oder einfach \tl (ohne Lösung im Quellcode)
\newcommand*{\truefalse}[1][{}]{% #1 Voreinstellung ist {}
\@ifundefined{ifprintanswers}{%
    \item[{\makebox[0pt][l]{$\square$}\hspace{2.5em}\makebox[0pt][r]{$\square$}}]
}{ 
\ifprintanswers% exam's Schalter für die Lösungen
    \ifthenelse{\equal{#1}{T}}{%
        \item[{\makebox[0pt][l]{$\checkmark$}\hspace{2.5em}\makebox[0pt][r]{$\square$}}]
    }{% sonst prüfe ob gleich F:
    \ifthenelse{\equal{#1}{F}}{%
        \item[{\makebox[0pt][l]{$\square$}\hspace{2.5em}\makebox[0pt][r]{$\checkmark$}}]
    }{% ansonsten ist eine Antwort nicht angegeben:
        \item[{\makebox[0pt][l]{$\square$}\hspace{2.5em}\makebox[0pt][r]{$\square$}}]
    }%
    }%
\else% sonst (sprich Antworten werden alle nicht gedruckt)
        \item[{\makebox[0pt][l]{$\square$}\hspace{2.5em}\makebox[0pt][r]{$\square$}}]
\fi
}
}% end newcommand


% Sub-Routinen für die Beschriftung der Spalten mit
% W  F (falls aufgerufen mit \begin{truefalses}[WF] )
% T  F (falls aufgerufen mit \begin{truefalses}[TF] )
% Mit dieser Definition darf die Beschriftung kein @ enthalten.
\def\extracttfletter@i#1#2@{\makebox[0pt][l]{\footnotesize{}#1}}
\def\extracttfletter@ii#1#2@{\makebox[0pt][r]{\footnotesize{}#2\,}}

% Umgebung für die Aufrufe von \truefalse:
\newenvironment{truefalses}[1][WF]{% #1 Voreinstellung ist "WF"
\vspace*{-1.2ex}% weniger vertikaler Abstand vor der ersten Zeile
\begin{list}{{}}{%
    \setlength{\leftmargin}{3.5em}% linke Einrückung
    \setlength{\labelwidth}{2.5em}% Länge der Beschriftung (vgl. \truefalse)
    % Abstand zwischen Beschriftung und Text:
    \setlength{\labelsep}{1em}% Voreinstellung ist .5em%
    }% Ende der Definition der Liste.
    % Füge eine Zeile ein mit der Beschriftung der Spalten für die Kreuze:
    \item[{\extracttfletter@i#1@{}%
        \hspace{2.5em}%
        \extracttfletter@ii#1@{}%
        }]%
        \phantom{.}% Die Zeile darf nicht leer sein.
    \vspace*{-1ex}% weniger vertikaler Abstand nach der Beschriftung der Spalten.
}{%
    \end{list}
}% Ende Definition der Umgebung 'truefalses'.




%% Seitenlayout (mit 'exam') --------
\extrawidth{1.6cm}
\extraheadheight[-1.5cm]{-1.2cm}% optionales Argument trifft auf erste Seite zu.
\extrafootheight{-1.2cm}

% oder alternativ stattdessen Seitenlayout mit 'geometry'
% \usepackage{geometry}
% \geometry{%
% 	margin=1.5cm
% }

\setlength{\parindent}{0pt}

\begin{document}
\pagestyle{empty}

% Schalter zum Drucken von Lösungen --------
% kommentieren, um Lösungen zu unterdrücken:
\printanswers
% ------------------------------------------

% Überschrift
\section*{Probeaufgaben, Mathematik 8b}

% Tabelle mit den Punkten:
\gradetable[h][questions]\\[.5ex]

% einleitender Text:
Übersichtlichkeit, Darstellung und Rechtschreibung werden bewertet.
Rechnungen müssen nach\-voll\-zieh\-bar gestaltet werden. 
Textaufgaben sollen mit einem Antwortsatz beendet werden. 
Achte bei allen Größen auf die richtige Einheit.



\subsection*{Aufgaben zum Thema Wurzeln}

\begin{questions}% Fragen --------


\question[2]
	Erkläre den Begriff der Quadratwurzel anhand des Beispiels $\sqrt{25}$.
	Nutze dazu die Fachsprache.
% Ende der Aufgabe.

\question[4]
	Charakterisiere die folgenden Zahlen, indem du die links aufgeführten Eigenschaften jeweils in der Tabelle abhakst:
\begin{table}[htpb]
\centering
\renewcommand{\arraystretch}{1.4}
\begin{tabular}{l*{4}{p{2cm}}}
\hline
	{Die Zahl ist\ldots}	& $-4$	& $\frac{72}{12}$	& $\sqrt{11}$ 	& $\sqrt{169}$	\\
\hline
	natürlich	\\
	negativ		\\ 
\hline
	ganz		\\
	rational	\\
\hline
	irrational	\\
	reell		\\
\hline
\end{tabular}
\end{table}
% Ende der Aufgabe mit der Tabelle.



\question[3]
	Berechne geschickt:
\begin{equation}
\aaa{}\quad \sqrt{2}\cdot \sqrt{18}			\qquad \qquad
\aaa{}\quad \sqrt{10}\cdot \sqrt{{3,6}}		\qquad \qquad
\aaa{}\quad \sqrt{125} : \sqrt{5}
\end{equation}
% Ende der Aufgabe.


\subsection*{Aufgaben zum Thema Wahrscheinlichkeitsrechnung}

\question[5]
	Aus einer Tüte mit 
	3 orangenen und 2 gelben Bonbons
%	2 orangenen und 3 gelben Bonbons
	wird zweimal je ein Bonbon gezogen und nicht zurückgelegt.
\begin{subparts}
	\subpart\label{sbp:baumdiagrammzeichnen1} Zeichne ein vollständiges Baumdiagramm.
\uplevel{Nutze das Baumdiagramm aus Teil~\ref{sbp:baumdiagrammzeichnen1} zur Beantwortung der folgenden Frage:}% end uplevel.
	\subpart Berechne die Wahrscheinlichkeit zwei verschiedenfarbige Bonbons zu ziehen.
\end{subparts}
% Ende der Aufgabe.

% Lösung zur Aufgabe oben:
\begin{solution}
\begin{subparts}
	\subpart Baumdiagramm:

	\includegraphics{baumdiagramm}

	\subpart Zu verschiedenfarbig gehören die Fälle von orange-gelb und gelb-orange. In beiden Fällen ist die Wahrscheinlichkeit gleich $\frac{3}{10}$, insgesamt also $\frac{6}{10}$.
\end{subparts}

\end{solution}









\question[5] Ein paar Teilaufgaben
\begin{parts}
	\part\label{prt:einteilvonxyz} asdf
\uplevel{Der Versuch, sich auf Teil~\ref{prt:einteilvonxyz} zu beziehen. Im Vergleich zu oben sollen nun andere Voraussetzungen gelten.}
	\part fds
	\part sd
	\begin{subsubparts}
		\subsubpart Version 1
		\subsubpart Version 2
		\subsubpart\label{ssp:dieseunteraufg} Version 3
	\end{subsubparts}
	wie in Teil~\ref{ssp:dieseunteraufg} bereits
\end{parts}


\question[3]
Kreuze jeweils an, ob die Aussage wahr (W) oder falsch (F) ist:
\begin{truefalses}
    \truefalse Erste Aussage (ohne hinterlegte Lösung im Quellcode)
    \truefalse[T] Die zweite Aussage dieser wahr/falsch-Ankreuzaufgabe enthält eine Gleichung:
    \begin{equation}
        \sum
    \end{equation}
    über die eine Aussage gemacht wird.
    \truefalse[F] Die dritte Aussage enthält so viel Text, dass der Text in die nachfolgende Zeile gedruckt werden muss.
\end{truefalses}


\end{questions}% Ende der Fragen --------


\clearpage
\addtocounter{page}{-1}
\thispagestyle{empty} 
\fillwithgrid{\stretch{1}}  % Kästchen

%% Eine leere Seite mit Zeilen/Kästchen zum Schreiben ---
%\clearpage
%\addtocounter{page}{-1}
%\thispagestyle{empty} 
%\fillwithgrid{\stretch{1}}  % Kästchen
%\fillwithlines{\stretch{1}} % Zeilen

%% Füge N cm leerer Zeilen/Kästchen zum Schreiben ein ---
% \fillwithgrid{Ncm}  % Kästchen, z.B. 5cm
% \fillwithlines{Ncm} % Zeilen, z.B. 5cm

%% Füge Zeilen/Kästchen bis zum Seitenende ein ---
% \fillwithgrid{\stretch{1}}  % Kästchen
% \fillwithlines{\stretch{1}} % Zeilen

\end{document}