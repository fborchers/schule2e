\documentclass{../exam2e}

%% Seitenlayout (mit 'exam') --------
\extrawidth{.9in}
\extraheadheight[-0.8in]{-.7in}% optionales Arg. trifft auf erste Seite zu.
\extrafootheight{-.7in}

% oder alternativ stattdessen Seitenlayout mit 'geometry'
% \usepackage{geometry}
% \geometry{%
% 	margin=1.5cm
% }

\setlength{\parindent}{0pt}

\begin{document}


\textbf{\large Klassenarbeit Nr. 1}\\[.5ex]


\gradetable[h][questions]

asdf

\begin{questions}% Fragen --------

\begin{question}[1]
	Eine Frage mit der Nummer \thequestion
\begin{parts}
	\part Teilaufgabe \\ Lorem ipsum\ldots
	\begin{subparts}
		\subpart Unteraufgabe mit Nummer
	\end{subparts}
	\part Nächste Teilaufgabe
\begin{subparts}
	\subpart Berechnen Sie.
	\subpart Berechnen Sie dazu außerdem:
\begin{subsubparts}
		\subsubpart lorem ipsum
		\subsubpart lorem ipsum
\end{subsubparts}
\end{subparts}
\end{parts}
\end{question}% Ende von Aufgabe 1

\question[2]\label{ques:eineaufg-zaehler}
Eine zweite Frage. Die Unteraufgaben können auch in einer mathematischen Zeile untergebracht werden:
\begin{equation}
	\aaa\quad \frac{1}{2}	\qquad\qquad\qquad
	\aaa\quad \frac{1}{2}	\qquad\qquad\qquad
	\aaa\quad \frac{1}{3}
\end{equation}



\renewcommand{\thesubpart}{\arabic{subpart}}% Unteraufgaben nummeriert
\renewcommand{\subpartlabel}{\thequestion.\thepartno.\thesubpart}% als: 1.1.1

\question[3] 
Für die Abituraufgaben kann der Zähler auch angepasst werden wie in dieser Aufgabe: 
\begin{parts}
	\part Teilaufgabe
\begin{subparts}
	\subpart\label{subpart:einpart} 
	Unteraufgabe
\end{subparts}
\end{parts}

Durch Label wie hier in Aufgabe~\ref{ques:eineaufg-zaehler} 
können die Aufgaben auch referenziert werden. 

\end{questions}% Ende der Fragen --------


\clearpage
\addtocounter{page}{-1}
\thispagestyle{empty} 
\fillwithgrid{\stretch{1}}  % Kästchen

%% Eine leere Seite mit Zeilen/Kästchen zum Schreiben ---
%\clearpage
%\addtocounter{page}{-1}
%\thispagestyle{empty} 
%\fillwithgrid{\stretch{1}}  % Kästchen
%\fillwithlines{\stretch{1}} % Zeilen

%% Füge N cm leerer Zeilen/Kästchen zum Schreiben ein ---
% \fillwithgrid{Ncm}  % Kästchen, z.B. 5cm
% \fillwithlines{Ncm} % Zeilen, z.B. 5cm

%% Füge Zeilen/Kästchen bis zum Seitenende ein ---
% \fillwithgrid{\stretch{1}}  % Kästchen
% \fillwithlines{\stretch{1}} % Zeilen

\end{document}