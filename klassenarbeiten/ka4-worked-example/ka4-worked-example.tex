\documentclass[12pt,fleqn,a4paper]{../exam2e}
% Mit 'fleqn' werden die mathematischen Gleichungen linksbündig.

\usepackage[ngerman]{babel}
\usepackage[german=quotes]{csquotes}

\usepackage{../gess}
\usepackage{../../mathe2e}

%% Lade PSTricks ein für die Vektorgeometrie ---
\usepackage{pst-3dplot}
%% pstricks settings for three dimensional coordinates
\psset{coorType=2}
%\psset{linecolor=black}% without effect on coord. systems.
% Labels of the coordinate axes: 
\psset{nameX=$x_1$,nameY=$x_2$,nameZ=$x_3$}
\psset{unit=1cm}


%% Seitenlayout (mit exam-Dokumentenklasse) --- 
\extrawidth{1.59cm}%
% Kopfzeile:
\extraheadheight[-2.3cm]{-1.9cm}%
% Fußzeile:
\extrafootheight[-2.2cm]{-1.8cm}%

% oder alternativ stattdessen Seitenlayout mit 'geometry'
% \usepackage{geometry}
% \geometry{%
% 	margin=1.5cm
% }



\begin{document}
\thispagestyle{empty}% erste Seite

% Dokumentkopf --------
%\klassenarbeitszeile{Klassenarbeit 1}{Mathematik}{10c}{10.10.2023}
\begin{klassenarbeitskopf}{Klassenarbeit 1}{Mathematik}{10c}{10.10.2024}
	Thema: Vektorgeometrie\\
	Zeit: 90 Minuten --- Hilfsmittel: Taschenrechner, Formelsammlung.
\end{klassenarbeitskopf}

\vspace*{1.5ex}
% einleitender Text:
Übersichtlichkeit, Darstellung und Rechtschreibung werden bewertet.
Rechnungen müssen nachvollziehbar gestaltet werden. 
%Textaufgaben sollen mit einem Antwortsatz beendet werden. 
Achte bei allen Größen auf die richtige Einheit.




\begin{questions}% Fragen --------

\subsection*{Teil 1: Keine Hilfsmittel zugelassen}

%% Aufg. 1 --------
\begin{question}[4]
	Löse das folgende lineare Gleichungssystem:
% TeX Ausdruck des Gleichungssystems : 
\begin{equation*}
\begin{lgs}{2}{1}
	&  & 3x& -& 7y	&==&  & 5	&	\\
	& -& 2x& +& 4y	&==& -& 7	&
\end{lgs}
\end{equation*}
\end{question}
\omitsolution

\fillwithgrid{\stretch{1}}  % Karos



\clearpage



%% Aufgabe 2 -- Punkte einzeichnen --------
\begin{question}[6]
Im abgebildeten Koordinatensystem befinden sich
\begin{itemize}
	\item der Punkt $P$ in der $x_1x_2$-Ebene,
	\item der Punkt $Q$ in der $x_2x_3$-Ebene,
	\item der Punkt $R$ in der $x_1x_3$-Ebene.
\end{itemize}
\begin{subparts}
	\subpart Bestimme die Koordinaten der drei Punkte $P$, $Q$ und $R$.
	\subpart Gib den Ortsvektor $\overrightarrow{OP}$ an und zeichne ihn ein.
	\subpart Zeichne den Punkt~$M\coords(-5,2,-4)$ ein. 
\end{subparts}

\begin{figure}
\psset{unit=1cm}
\begin{pspicture}(-4,-3)(5,4)
\psgrid[unit=0.5cm,subgriddiv=1,gridlabels=0pt,gridwidth=0.2pt,gridcolor=gray](-8,-5)(10.5,8)
\pstThreeDCoor[IIIDticks,IIIDlabels,xMin=-5.5,xMax=4,yMin=-4,yMax=4,zMin=-2,zMax=3]
\pstThreeDDot(3,2,0)
\pstThreeDPut(3,2.3,.2){$P$}
\pstThreeDDot(0,1.5,2)
\pstThreeDPut(0,1.2,2.2){$Q$}
\pstThreeDDot(3,0,4)
\pstThreeDPut(3,-.3,4.2){$R$}
\end{pspicture}	
\end{figure}
\end{question}

\begin{solution}
Die Koordinaten lauten: $P\coords(3,2,0)$, $Q\coords(0,\frac32,2)$ 
und $R\coords(3,0,4)$.

\begin{pspicture}(-4,-2.5)(5,3.8)
\psgrid[unit=0.5cm,subgriddiv=1,gridlabels=0pt,gridwidth=0.2pt,gridcolor=gray](-8,-4)(10.5,7)
\pstThreeDCoor[IIIDticks,IIIDlabels,xMin=-5.5,xMax=4,yMin=-4,yMax=4,zMin=-2,zMax=3]
\pstThreeDDot[drawCoor=true](3,2,0)
\pstThreeDPut(3,2.3,.2){$P$}
\pstThreeDDot[drawCoor=true](0,1.5,2)
\pstThreeDPut(0,1.2,2.2){$Q$}
\pstThreeDDot[drawCoor=true](3,0,4)
\pstThreeDPut(3,-.3,4.2){$R$}
\pstThreeDDot[drawCoor=true](-5,2,-4)
\pstThreeDPut(-5,1.8,-4.2){$M$}
\end{pspicture}
\end{solution}
% end aufgabe punkte im raum. 






%% Aufgabe 3 --------
% Widerlege die Aussage (als AFB3), ursprünglich waren dies Ankreuzaufgaben.
\begin{question}[4] 
Widerlege die folgenden Aussagen:
\begin{subparts}
	\subpart Wenn bei einem Vektor zwei Komponenten negativ sind, dann müssen bei einem anderen, parallelen Vektor diese Komponenten auch negativ sein. 
	\subpart Wenn zwei Geraden einen gemeinsamen Spurpunkt haben, dann sind sie parallel oder identisch.
\end{subparts}
\end{question}
\omitsolution

\fillwithgrid{\stretch{1}}  % Karos









\clearpage 
\subsection*{Teil 2: mit Taschenrechner und Formelsammlung}



%% Aufgabe mit der Pyramide, deren Grundfläche gedreht ist.
\begin{question}[7]
	Es ist eine Pyramide mit quadratischer Grundfläche~$ABCD$ und Spitze~$S$ gegeben.

\begin{figure}%% Abbildung einer quadratischen Pyramide ---
\begin{minipage}[t]{0.45\textwidth}\vspace{0pt}
\centering\small
\psset{unit=1cm}
\begin{pspicture}(-3,-1.2)(3,5.4)
%\psframe[linecolor=yellow!90!black](-3,-1.2)(3,5.4)
%	\pstThreeDCoor[axes=none]
	% Punkte der quadratischen Grundfläche:
	\pstThreeDNode(1.5,-1.5,0){A}\uput[dl](A){$A$}
	\pstThreeDNode(1.5,1.5,0){B}\uput[dr](B){$B$}
	\pstThreeDNode(-1.5,1.5,0){C}\uput[r](C){$C$}
	\pstThreeDNode(-1.5,-1.5,0){D}\uput[l](D){$D$}
	% Spitze:
	\pstThreeDNode(0,0,5){S}\uput[ur](S){$S$}
	\pspolygon[linestyle=none,fillstyle=solid,fillcolor=gray!20](A)(B)(C)(D)(A)
	\psline(B)(S)(C)(B)(A)(S)% Ränder
	\psset{linestyle=dashed}%Linie hinten gestrichelt
	\psline(C)(D)(A)
	\psline(S)(D)
\end{pspicture}
\end{minipage}\hfill
\begin{minipage}[t]{0.5\textwidth}\vspace{0pt}%
	\caption{Nicht-maß\-stäbliche Abbildung einer Pyramide mit quadratischer Grundfläche}
\end{minipage}
% c.f. this construction with minipages is "direct and robust":
% https://tex.stackexchange.com/a/29163/154857
\end{figure}

\noindent Von der Grundfläche sind die folgenden drei Punkte bekannt:
\begin{equation*}
	A\coords(-5,-3,0), 	\quad
	C\coords(12,4,0) 	\quad\text{und}\quad
	D\coords(0,9,0).
\end{equation*}%
% Beispiel für die Verwendung von siunitx's \num{...} Befehl:
Die Spitze~$S$ liegt bei $S\coords(\num{3.5}, \num{.5}, 5)$.
\begin{subparts}
	\subpart Bestimme die Koordinaten des Eckpunktes~$B$.% so, dass $ABCD$ ein Quadrat ergibt.
	\subpart Berechne den Abstand des Mittelpunktes~$M_{BC}$ der Seite~$BC$ von der Spitze~$S$ der Pyramide.
\end{subparts}
\end{question}
\omitsolution










%% Einfache Fragen zum Schnittpunkt ---
\begin{question}[7]
%	Gegeben sind die Geraden $g$ und $h$ mit
Untersuche die Lagebeziehung der folgenden Geraden zueinander und bestimme gegebenenfalls den Schnittpunkt.
	\begin{equation*}
		g:\;\vec{x} = \vector(\frac27,1,0) + t\cdot \vector(-3,-3,-4)
		\qquad\text{und}\quad
		h:\;\vec{x} = \vector(0,9,1) + s \cdot \vector(-5,-7,-7)
	\end{equation*}
\end{question}








% Vergleichbare Anwendungsaufgabe zu Geraden: Zwei Flugzeuge.
\begin{question}[19]
Zwei Flugzeuge fliegen mit je konstanter Geschwindigkeit auf gradlinigen Flugbahnen. Die Position der Flugzeuge wird bezüglich eines Koordinatensystems mit der Längeneinheit \qty{1}{\km} angegeben, die $x_3$-Koordinate gibt die Flughöhe an. 
Um 8:00 Uhr ist das Flugzeug~1 im Punkt $P_1\coords(-10,0,0)$
und das Flugzeug~2 im Punkt $P_2\coords(-25,-30,8)$. 

Wir betrachten im folgenden die Zeit $t$ in \unit{\min} ab 8:00 Uhr:
Nach vier Minuten hat das Flugzeug~1 die Position $Q_1\coords(6,16,4)$ erreicht. 
Nach fünf Minuten hat das Flugzeug~2 die Position $Q_2\coords(20,30,8)$ erreicht. 

\begin{parts}
% Geradengleichung erläutern.
\part\label{prt:gleichflugz1} Erläutere, warum die Gleichung
\begin{equation}
	g_1:\quad \overrightarrow{OX} = \vector(-10,0,0) + t\cdot\vector(4,4,1)
	\tag{*}
	\label{eq:bwglflugzeug1}
\end{equation}
die Position des Flugzeugs~1 in Abhängigkeit von der Zeit angibt.

% Geradengleichung ermitteln:
\part Ermittle analog zu Gleichung~\eqref{eq:bwglflugzeug1} in Teilaufgabe~\ref{prt:gleichflugz1}
eine Gleichung~$g_2$, die die Position des Flug\-zeugs~2 in Abhängigkeit von der Zeit angibt.

\part Berechne die Geschwindigkeit des Flugzeugs~1 in \unit{\km\per\hour}.

\uplevel{Der Luftraum, der von den Flugzeugen genutzt wird, wird von zwei verschiedenen Radarstationen überwacht. 
%Die \glqq{}Übergabe\grqq %% requires babel, or with shorthands:
Die "`Übergabe"' der Flugzeuge erfolgt, 
wenn die Flugzeuge die $x_2x_3$-Ebene durchfliegen.}
	\part Bestimme den Zeitpunkt und die Positions des Flugzeugs~1 bei der Übergabe. 

	\part Ermittle, zu welchem Zeitpunkt das Flugzeug~1 eine Flughöhe von \qty{8}{\km} erreicht und wie groß zu diesem Zeitpunkt der Abstand der beiden Flugzeuge ist. 
\end{parts}
\end{question}
\omitsolution











%% Aufgabe Parameter: Beurteile Existenz
\begin{question}[4]
	Gegeben sei ein Vektor $\vec{v}$ mit einem noch unbestimmten Eintrag $a\in \setR$: 
\begin{equation*}
	\vec{v} = \vector(17,a,8)
\end{equation*}
	Beurteile, ob es Werte~$a$ gibt, für die der Vektor~$\vec{v}$ die Länge~$18$ hat.
\end{question}
\omitsolution






\end{questions}% Ende der Fragen --------


%% Eine leere Seite mit Zeilen/Kästchen zum Schreiben ---
%\clearpage
%\addtocounter{page}{-1}
%\thispagestyle{empty} 
%\fillwithgrid{\stretch{1}}  % Kästchen
%\fillwithlines{\stretch{1}} % Zeilen

%% Füge N cm leerer Zeilen/Kästchen zum Schreiben ein ---
% \fillwithgrid{Ncm}  % Kästchen, z.B. 5cm
% \fillwithlines{Ncm} % Zeilen, z.B. 5cm

%% Füge Zeilen/Kästchen bis zum Seitenende ein ---
% \fillwithgrid{\stretch{1}}  % Kästchen
% \fillwithlines{\stretch{1}} % Zeilen

\end{document}