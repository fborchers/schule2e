\documentclass{../exam2e}

% ka1-minimal ist ein Minimalbeispiel, das die Struktur der Aufgaben und Teilaufgaben zeigt. 
\begin{document}

\gradetable[h][questions]


\begin{questions}% Fragen --------

\question[1]
	Eine Frage mit der Nummer \thequestion
\begin{parts}
	\part Teilaufgabe \\ Lorem ipsum\ldots
	\begin{subparts}
		\subpart Unteraufgabe mit Nummer
	\end{subparts}
	\part Nächste Teilaufgabe
\end{parts}
% Ende von Aufgabe 1

\question[2] Eine zweite Frage. Hier folgt als nächste Struktur nicht die Teilaufgabe, sondern gleich die Unteraufgabe. Die Abstände werden automatisch angepasst.
\begin{subparts}
		\subpart Unteraufgabe. Untersuchen Sie  dazu:
	\begin{subsubparts}
		\subsubpart lorem ipsum
		\subsubpart lorem ipsum
	\end{subsubparts}
\end{subparts}
% Ende der Aufgabe.

\clearpage 

\question[3] Diese Aufgabe zeigt die ganze Struktur der documentclass \texttt{exam}. 
\begin{parts}
	\part Teilaufgabe
\begin{subparts}
	\subpart Unteraufgabe
	\begin{subsubparts}
		\subsubpart Unterunteraufgabe
	\begin{checkboxes}
		\choice Ankreuzmöglichkeit
	\end{checkboxes}
	\end{subsubparts}
\end{subparts}
\end{parts}
% Ende der Aufgabe.


\end{questions}% Ende der Fragen --------

\end{document}
