\documentclass[12pt,fleqn,a4paper]{../exam2e}

\usepackage{graphicx}% Bilder

\usepackage[ngerman]{babel}% Sprache

\usepackage{cmbright}% Schriftart
\usepackage{../../mathe2e}% Mathematiksatz


% Seitenlayout mit 'geometry'
\usepackage{geometry}
\geometry{%
	top=.8cm,
	margin=1.5cm,
	right=2.2cm,
	bottom=2cm
}


% Einrückungen ---
\setlength{\parindent}{0pt}% Paragraphen links
\raggedright% Flattersatz rechts

% Unteraufgaben ganz links einrücken:
\renewcommand{\subpartshook}{% 
  \setlength{\leftmargin}{0pt}%
  \setlength{\labelwidth}{-\labelsep}%
}




% Formatierung der Aufgaben --------

% Nenne Punkte hier "Bewertungseinheiten" BEs:
\pointpoints{BE}{BE}
\marginpointname{ BE}

% Drucke die Punkte von Teilaufgaben rechts auf den Rand:
\pointsdroppedatright
% Verwende dazu je \droppoints.

% Die Unteraufgaben sollen nummeriert sein:
\renewcommand{\thesubpart}{\thepartno.\arabic{subpart}}% 1.1.1


% Aufgabenlayout ---
% vgl. https://tex.stackexchange.com/questions/111460/how-do-i-use-titledquestion-in-the-exam-class-with-or-without-points
\makeatletter
\qformat{%
    {\bfseries Aufgabe \thequestion: \thequestiontitle}% 
    \enspace\if\totalpoints0\else~(\totalpoints~\points)\fi%
    \hfill%
    }
\makeatother





\begin{document}
\section*{Abiturprüfung 2023:}

\begin{questions}% Aufgaben --------

%% Aufgabe 1
\titledquestion{Titel der ersten Aufgabe}
Für die Abituraufgaben können Nummerierung und Punkte auch angepasst werden wie etwa in dieser Aufgabe.
\begin{parts}
\part[1] Teilaufgabe tue jenes\ldots\droppoints
\begin{subparts}
	\subpart[2]\label{ssp:dieseeineunteraufgabe} Unteraufgabe
	\begin{equation*}
		\frac{2\pi}{T}
	\end{equation*}
	Löse\ldots\droppoints
	\subpart[1] Gib an\ldots\droppoints
\end{subparts}
	\part[10] Weitere Teilaufgabe. Die Punkte werden mit dem Befehl \textbackslash droppoints am Ende des Absatzes rechts auf den Rand gedruckt. 
	\droppoints
\end{parts}
% Ende Aufgabe

%% Aufgabe 2
% Zweite Aufgabe ---
\titledquestion{Titel der zweiten Aufgabe}\label{que:dieeineaufg}
\begin{parts}
    \part[10] usw.\droppoints
    \part[10] usw.\droppoints
\end{parts}
% Ende Aufgabe



\titledquestion{Titel dieser Aufgabe}
\begin{parts}
	\part[5] Die folgende Zeile zeigt, wie die Punkte auf den Rand gedruckt werden, wenn die letzte Zeile dieser Teilaufgaben eine Gleichung ist:
	% siehe: https://tex.stackexchange.com/questions/452064/how-to-place-point-count-information-on-the-line-of-a-displayed-equation-in-exam
\begin{equation}
	a^ab^b\ge\left(\frac{a+b}{2}\right)^{\!a+b}\ge a^bb^a\,.
	\tag*{\parbox{1pt}{\droppoints}}% accident
\end{equation}
	\part[1] Mit \textbackslash{}label und \textbackslash{}ref können die Aufgaben wie üblich referenziert werden, wie hier z.B. die Aufgabe~\ref{que:dieeineaufg} oder die Unteraufgabe~\ref{ssp:dieseeineunteraufgabe}.\droppoints
\end{parts}
% Ende Aufgabe.




\end{questions}

\end{document}
