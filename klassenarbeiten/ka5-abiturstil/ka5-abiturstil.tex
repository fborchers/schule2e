\documentclass[12pt,fleqn,a4paper]{../exam2e}
% Mit 'fleqn' werden die mathematischen Gleichungen linksbündig.

\usepackage{graphicx}% Bilder laden

%% Spracheinstellung für Deutsch
\usepackage[ngerman]{babel}
\usepackage[german=quotes]{csquotes}
%% oder Englisch
%\usepackage[ngerman,english]{babel}
%\usepackage{csquotes}


\usepackage{../abitur2e}% spezielle Formatierung für das Abitur
\usepackage{../../mathe2e}% Mathematiksatz




% Schalter zum Drucken von Lösungen --------
% kommentieren, um Lösungen zu unterdrücken:
\printanswers
%\unframedsolutions% falls kein Rand gewünscht.
% ------------------------------------------

\begin{document}
%\section*{Abiturprüfung 2027}

\begin{questions}% Aufgaben --------



%% Aufgabe 1
\titledquestion{Titel der ersten Aufgabe}
Für die Abituraufgaben können Nummerierung und Punkte auch angepasst werden wie etwa in dieser Aufgabe.
\begin{parts}
\part[1] Teilaufgabe tue jenes\ldots\droppoints
\begin{subparts}
	\subpart[2]\label{ssp:dieseeineunteraufgabe} Unteraufgabe
	\begin{equation*}
		\frac{2\pi}{T}
	\end{equation*}
	Löse\ldots\droppoints
	\subpart[3] Gib an in den folgenden Fällen\droppoints
\begin{subsubparts}
	\subsubpart eins
	\subsubpart zwei
	\subsubpart drei
\end{subsubparts}
\end{subparts}
	\part[10] Weitere Teilaufgabe. Die Punkte werden mit dem Befehl \textbackslash droppoints am Ende des Absatzes rechts auf den Rand gedruckt. 
	\droppoints
\end{parts}
% Ende Aufgabe



\begin{solution}
Die Aufgabe ist eigentlich ganz einfach:
%	asdf \marginpar{asdf}
%	asdf \marginnote{1BE}
\begin{parts}
	\part Teil eins\ldots \afbi{1}\\
	Außerdem\afbii{2}\\
	Außerdem\afbiii{2}
\end{parts}
\end{solution}% Aufgabe 1


\titledquestion{Titel dieser Aufgabe}
\begin{parts}
	\part[5] Die folgende Zeile zeigt, wie die Punkte auf den Rand gedruckt werden, wenn die letzte Zeile dieser Teilaufgaben eine Gleichung ist:\label{que:dieeineaufg}
	% siehe: https://tex.stackexchange.com/questions/452064/how-to-place-point-count-information-on-the-line-of-a-displayed-equation-in-exam
\begin{equation}
	a^ab^b\ge\left(\frac{a+b}{2}\right)^{\!a+b}\ge a^bb^a\,.
	\tag*{\parbox{1pt}{\droppoints}}% accident
\end{equation}
	\part[1] Mit \textbackslash{}label und \textbackslash{}ref können die Aufgaben wie üblich referenziert werden, wie hier z.B. die Aufgabe~\ref{que:dieeineaufg}.\droppoints
\end{parts}


% Aufgabe 2


\end{questions}

\end{document}
