\documentclass[a4paper,12pt]{../exam2e}

\usepackage{graphicx}

\usepackage[ngerman]{babel}

\usepackage{cmbright}



%% Seitenlayout (mit 'exam') --------
\extrawidth{1.4cm}
\extraheadheight[-1.5cm]{-1.2cm}% optionales Argument trifft auf erste Seite zu.
\extrafootheight{-1.2cm}

% oder alternativ stattdessen Seitenlayout mit 'geometry'
% \usepackage{geometry}
% \geometry{%
% 	margin=1.5cm
% }

\setlength{\parindent}{0pt}


\begin{document}

\begin{questions}% Fragen --------



\pointsdroppedatright
\renewcommand{\thesubpart}{\arabic{subpart}}% Unteraufgaben nummeriert
\renewcommand{\subpartlabel}{\thequestion.\thepartno.\thesubpart}% als: 1.1.1
\qformat{\bfseries Aufgabe~\thequestion\hfill}

\question[1]
Für die Abituraufgaben kann der Zähler und Punkte auch angepasst werden wie in dieser Aufgabe: 
\begin{parts}
\part[1] Teilaufgabe
\begin{subparts}
	\subpart Unteraufgabe
\end{subparts}
	\part[2] Weitere Teilaufgabe. Die Punkte werden mit dem Befehl \textbackslash droppoints am Ende des Absatzes gedruckt. 
	\droppoints
\end{parts}

\end{questions}
\end{document}