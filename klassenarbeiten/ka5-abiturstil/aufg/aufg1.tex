

%% Aufgabe 1
\titledquestion{Titel der ersten Aufgabe}
Für die Abituraufgaben können Nummerierung und Punkte auch angepasst werden wie etwa in dieser Aufgabe.
\begin{parts}
\part[1] Teilaufgabe tue jenes\ldots\droppoints
\begin{subparts}
	\subpart[2]\label{ssp:dieseeineunteraufgabe} Unteraufgabe
	\begin{equation*}
		\frac{2\pi}{T}
	\end{equation*}
	Löse\ldots\droppoints
	\subpart[3] Gib an in den folgenden Fällen\droppoints
\begin{subsubparts}
	\subsubpart eins
	\subsubpart zwei
	\subsubpart drei
\end{subsubparts}
\end{subparts}
\begin{EnvUplevel}
	Man kann mit dem Befehl \verb+\uplevel+ bzw.\ der Umgebung \verb+EnvUplevel+ erreichen, dass der Text gedruckt wird mit der Formatierung, wie sie eine Ebene oberhalb der aktuellen gilt. Dieser Text hier ist formatiert wie der Text der Aufgabe, % (in der exam-Terminologie \verb+question+), 
	es folgt aber die nächste Teilaufgabe.
\end{EnvUplevel}
	\part[10] Weitere Teilaufgabe. Die Punkte werden mit dem Befehl \textbackslash droppoints am Ende des Absatzes rechts auf den Rand gedruckt. 
	\droppoints
\end{parts}
% Ende Aufgabe



\begin{solution}
Die Aufgabe ist eigentlich ganz einfach:
%	asdf \marginpar{asdf}
%	asdf \marginnote{1BE}
\begin{parts}
	\part Teil eins\ldots \afbi{1}\\
	Außerdem\afbi{1}\afbii{2}\afbiii{3}\\
	Außerdem\afbiii{2}
\end{parts}
\end{solution}