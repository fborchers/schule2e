
\titledquestion{Titel dieser Aufgabe}
\begin{parts}
	\part[5] Die folgende Zeile zeigt, wie die Punkte auf den Rand gedruckt werden, wenn die letzte Zeile dieser Teilaufgaben eine Gleichung ist:\label{que:dieeineaufg}
	% siehe: https://tex.stackexchange.com/questions/452064/how-to-place-point-count-information-on-the-line-of-a-displayed-equation-in-exam
\begin{equation}
	a^ab^b\ge\left(\frac{a+b}{2}\right)^{\!a+b}\ge a^bb^a\,.
	\tag*{\parbox{1pt}{\droppoints}}% accident
\end{equation}
	\part[1] Mit \textbackslash{}label und \textbackslash{}ref können die Aufgaben wie üblich referenziert werden, wie hier z.B. die Aufgabe~\ref{que:dieeineaufg}.\droppoints
\end{parts}


