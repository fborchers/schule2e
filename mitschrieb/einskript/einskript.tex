\documentclass[a4paper,12pt]{article}


\usepackage{graphicx}
\usepackage[ngerman]{babel}

\usepackage{../../exam2e}% lädt unser exam2e.sty
\usepackage{../../mathe2e}% lädt unser math2e.sty


\usepackage{hyperref}

\setlength{\parindent}{1em}
\setlength{\parskip}{0pt}


\begin{document}

\section{Ein Skript}
In den Skripten kommt nicht die Klasse \texttt{exam} zum Einsatz, sondern nur der minimale Aufsatz \texttt{exam2e} mit den Auflistungen Fragen, Teilaufgaben, Unteraufgaben und Unterunteraufgaben. Dies dient der Kompatibilität des Aufgabenformates von \texttt{exam} mit größeren Dokumentenklassen wie \texttt{article} oder \texttt{book}.

Die Struktur von \texttt{exam} kann auch außerhalb der Hierarchie verwendet werden. Hier etwa stehen die Auflistungen von \texttt{subpart} nicht im Rahmen einer Frage:
\begin{subparts}
	\subpart Unteraufgabe

\uplevel{Dieser Text hier steht mithilfe von \texttt{\textbackslash{}uplevel} wieder außerhalb. Die nächste Unteraufgabe setzt die Nummerierung fort.}
	\subpart noch eine Unteraufgabe
	\subpart und noch eine Unteraufgabe
\end{subparts}

\subsection{Prüfungsaufgaben}
\question%\begin{question}
	Eine Frage mit der Nummer \thequestion
\begin{parts}
	\part Teilaufgabe

	Lorem ipsum\ldots
	\begin{subparts}
		\subpart Unteraufgabe mit Nummer, die so lang ist das ein Zeilenumbruch notwendig wird.

		Außerdem
	\end{subparts}
	\part Nächste Teilaufgabe lautet berechne:%
	\label{part:dieseeineteilaufgabe}
\begin{equation}
	\aaa\quad \frac{1}{2} \qquad\qquad
	\aaa\quad \frac{1}{2} 
\end{equation}
\end{parts}
%\end{question}% Ende von Aufgabe 1
%
\begin{solution}
	Antwort ...
\end{solution}


Text zwischen den Fragen\ldots Eine zweite Frage. Hier folgt als nächste Struktur nicht die Teilaufgabe, sondern gleich die Unteraufgabe. Die Abstände werden automatisch angepasst, die Einstellung des \texttt{parindent} gilt auch für Absätze innerhalb von Fragen.

Ref: wie man in Teil~\ref{part:dieseeineteilaufgabe} sehen kann.


\question%\begin{question}
Eine zweite Frage. Hier folgt als nächste Struktur nicht die Teilaufgabe, sondern gleich die Unteraufgabe. Dieses Vorgehen ist in der Unter- und Mittelstufe üblich.

Die Abstände werden automatisch angepasst.
\begin{subparts}
		\subpart\label{sbp:unteraufg1a} Unteraufgabe. Untersuchen Sie  dazu:
	\begin{subsubparts}
		\subsubpart lorem ipsum
		\subsubpart lorem ipsum
	\end{subsubparts}
		\subpart Neue Unteraufgabe, die sich von Unteraufg.~\ref{sbp:unteraufg1a} unterscheidet. Untersuchen Sie dazu nun:
	\begin{subsubparts}
		\subsubpart lorem ipsum
		\subsubpart lorem ipsum
		\subsubpart lorem ipsum
	\end{subsubparts}
\end{subparts}
%\end{question}
\omitsolution



\question%\begin{question}
	Diese Aufgabe zeigt die ganze Struktur der documentclass \texttt{exam2e}, also question, part, subpart, subsubpart und choice:
\begin{parts}
	\part Teilaufgabe
\begin{subparts}
	\subpart Unteraufgabe
\uplevel{Test von \textbackslash{}uplevel \ldots}
	\subpart Unteraufgabe
	\begin{subsubparts}
		\subsubpart Unterunteraufgabe
	\begin{checkboxes}
		\choice Ankreuzmöglichkeit (alias choice)
	\end{checkboxes}
	\end{subsubparts}
\end{subparts}
\uplevel{Test von \textbackslash{}uplevel \ldots}
	\part Und dann wählen Sie noch hier:
	\begin{checkboxes}
		\choice Ankreuzmöglichkeit 1 \\ mit einem Zeilenumbruch.
		\choice Ankreuzmöglichkeit 2
	\end{checkboxes}
\end{parts}
%\end{question}
\omitsolution




\question[5]%\begin{question}[5]
	Aus einer Tüte mit 
	3 orangenen und 2 gelben Bonbons
%	2 orangenen und 3 gelben Bonbons
	wird zweimal je ein Bonbon gezogen und nicht zurückgelegt.
\begin{subparts}
	\subpart Zeichne ein vollständiges Baumdiagramm.
\uplevel{Nutze das Baumdiagramm zur Beantwortung der folgenden Frage:}%
	\subpart Berechne die Wahrscheinlichkeit zwei verschiedenfarbige Bonbons zu ziehen.
\end{subparts}
%\end{question}

% Lösung zur Aufgabe oben:
\begin{solution}[Lösung der Aufgabe mit Bonbontüte]
\begin{subparts}
	\subpart Baumdiagramm:

	\includegraphics{baumdiagramm}

	\subpart Zu verschiedenfarbig gehören die Fälle von orange-gelb und gelb-orange. In beiden Fällen ist die Wahrscheinlichkeit gleich $\frac{3}{10}$, insgesamt also $\frac{6}{10}$.
\end{subparts}
\end{solution}






\end{document}




